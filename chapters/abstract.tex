% !TeX root = ../main.tex

\ustcsetup{
  keywords  = {高性能并行, Top-k, 国产AI处理器},
  keywords* = {dissertation, abstract, keywords},
}

\begin{abstract}
  伴随着深度学习技术的快速发展,Top-k 查询算法在深度学习的各个细分领域得到了广泛应用,成为许多任务中的核心操作。然而,深度学习模型中庞大的数据规模使传统 Top-k 查询算法在处理海量数据时效率低下,难以满足快速响应和精确查询的需求。为此,设计高性能并行的 Top-k 查询算法,并充分结合深度学习处理器的硬件特性优化数据处理效率,成为当前研究的重要方向。PuDianNao 系列处理器是专为深度学习应用设计的国产 AI 硬件架构,具有指令集灵活、并行计算能力强及硬件资源利用率高等特点。然而,现有算法多针对传统 CPU 或 GPU 环境优化,未能充分挖掘国产 AI 处理器的潜力。本文结合 PuDianNao 的硬件特性,设计并实现了两种高效的 Top-k 查询算法,并在深度学习任务中验证了其性能优势与应用效果。主要贡献如下:

1.基于 Radix-Select 的 Top-k 查询算法的设计与实现

本文设计了一种基于 Radix-Select 算法的 Top-k 查询方法,充分利用了 PuDianNao 处理器在排序操作中的硬件加速能力。通过优化桶操作的并行性、内存访问效率以及流水线指令,大幅提升算法吞吐量。实验结果表明,该算法在数据量较大时,性能相比传统 CPU 基于快速排序的 Top-k 算法提升了 20 倍以上,相较于 NVIDIA A100 GPU 也表现出性能优势,有效降低了查询延迟。

2.基于 Quick-Select 的 Top-k 查询算法的设计与实现

针对 PuDianNao 的多核架构,设计了一种基于 Quick-Select 的并行 Top-k 查询算法。Quick-Select 通过分区思想逐步缩小搜索范围,结合处理器的多核特性,设计了高效的负载均衡机制和任务分配策略,使算法能在多个计算核上并行执行。通过优化递归调用,减少了不必要的分区和内存操作,进一步提升查询效率。

3.基于深度学习模型的功能性验证

为验证算法的实际应用效果,本文结合 Pytorch 深度学习框架,将优化后的 Top-k 查询算法应用于 ResNet107 网络的目标检测任务。实验表明,在包含百万级候选区域的大规模数据集中,该算法显著提升了候选区域筛选的速度,同时保持了与基准方法相当的检测精度。

本文提出的基于 PuDianNao 系列处理器的高效 Top-k 查询算法,通过结合硬件特性进行定制化优化,显著提升了深度学习任务中 Top-k 操作的效率。为高效处理海量数据的 Top-k 查询问题提供了新的解决方案。

\end{abstract}

\begin{abstract*}
  This is a sample document of USTC thesis \LaTeX{} template for bachelor,
  master and doctor. The template is created by zepinglee and seisman, which
  orignate from the template created by ywg. The template meets the
  equirements of USTC thesis writing standards.

  This document will show the usage of basic commands provided by \LaTeX{} and
  some features provided by the template. For more information, please refer to
  the template document ustcthesis.pdf.
\end{abstract*}
