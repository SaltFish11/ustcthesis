\chapter{绪论}
\section{研究背景及意义}
近年来,随着深度学习和人工智能技术的快速发展,全球范围内对数据计算与处理能力的需求急剧上升。特别是在深度学习、推荐系统、搜索引擎和目标检测等应用场景中,Top-k 查询算法作为核心数据筛选工具,被广泛应用于从大规模数据集中选取最优数据。Top-k 查询的主要任务是从海量数据中快速筛选出具有最高优先级或得分的前 k 个数据元素,为后续的模型处理或决策提供支持。尤其是在深度学习模型中,Top-k 查询被频繁用于多个关键操作,例如筛选权重较大的神经元、从候选区域中选出优先级最高的目标框,以及在自然语言处理任务中选取概率最高的词汇或短语。它直接影响模型的运行效率和预测性能,是深度学习系统中不可或缺的重要组成部分。

然而,随着数据规模的指数增长和模型复杂度的不断提升,传统基于 CPU 和 GPU 的 Top-k 查询算法在计算性能、能耗比和响应时间等方面的局限性逐渐显现,难以满足当前深度学习与大数据处理场景中的高性能需求。Top-k 查询算法的核心挑战在于如何在海量数据中以更低的时间复杂度和计算成本完成快速排序和筛选操作。传统算法通常基于排序(如快速排序或堆排序)或分治策略(如快速选择算法),但这些方法在面对超大规模数据集时,往往计算代价高昂且对硬件资源的利用率不够高。此外,随着实时性要求的增加,例如推荐系统中的动态数据筛选、目标检测中的实时候选框生成,传统算法在处理延迟和能耗控制方面的劣势尤为突出。

与此同时,作为人工智能技术的重要支撑,AI 芯片的研发和应用近年来成为全球科技竞争的关键领域。AI 芯片是专为深度学习等人工智能任务设计的硬件加速器,它不同于传统的通用 CPU 或 GPU,而是针对特定任务进行了优化,能够在算力密集型任务中提供更高效的处理能力。例如,AI 芯片通常配备大规模并行计算单元和专用的硬件加速模块,支持高吞吐量的矩阵运算和深度学习推理计算。近年来,国产AI芯片作为我国推进科技自主可控、实现关键技术突破的重要组成部分,取得了快速发展。以寒武纪、华为昇腾、天数智芯等为代表的国产芯片在算力、能效比和算法支持等方面已接近国际一流水平。

然而,与国外成熟的 CPU 和 GPU 生态相比,国产AI芯片在算法优化和生态完善方面仍有较大的提升空间。许多现有算法的设计和实现主要针对通用 CPU 或 GPU 的硬件架构进行优化,例如利用 GPU 的多线程并行性或 CPU 的缓存特性。而国产AI芯片通常具有不同的硬件架构和指令集设计,例如多核异构计算、片上存储和流水线优化等,其性能潜力未被充分挖掘,导致部分任务在国产芯片上的性能表现并未达到最优。如何通过算法与硬件的深度协同优化,充分发挥国产AI芯片的硬件特性,解决传统方法在处理海量数据中的性能瓶颈,成为当前研究的重要方向。

在这一背景下,研究基于国产AI芯片的 Top-k 查询算法具有重要意义。首先,该研究有助于提升国产芯片的应用价值和市场竞争力。通过针对芯片架构特性的定制化优化,可以设计出更高效的 Top-k 查询算法,使国产AI芯片在深度学习和大数据处理任务中的性能表现更加突出。这将显著增强国产芯片的市场竞争力,推动其在人工智能相关领域的广泛应用,为我国芯片产业的发展提供技术支撑。

其次,该研究将推动深度学习和大数据处理领域的技术创新。Top-k 查询算法是多个关键任务中的基础操作,其性能直接影响整个系统的效率。通过结合国产AI芯片的硬件特点进行优化,可以大幅提升查询效率,降低数据处理延迟,为深度学习模型的训练和推理提供更好的支持。这一研究还将为其他领域的高性能算法设计提供参考,如推荐系统、搜索引擎和金融风控等需要实时数据筛选的场景。

再次,该研究对于实现科技自主可控、保障国家数据安全具有重要意义。在当前国际科技竞争加剧的背景下,数据处理与计算能力已成为衡量国家科技实力的重要指标。通过基于国产AI芯片的算法优化研究,可以逐步减少对国外硬件平台和技术生态的依赖,形成自主可控的技术体系,提升我国在人工智能领域的核心竞争力。同时,国产芯片在重要领域中的广泛应用,也将进一步保障我国的数据安全和技术主权。

最后,该研究还将为软硬件协同优化提供新的实践经验。AI 芯片的性能提升不仅依赖于硬件设计,还需要与上层算法和应用深度融合。通过针对国产AI芯片的硬件架构设计高效的 Top-k 查询算法,可以探索硬件资源的最佳使用方式,例如如何优化流水线执行效率、如何提高内存访问效率等。这一过程将为未来国产芯片的设计和优化积累宝贵经验,推动软硬件协同发展的创新。

综上所述,基于国产AI芯片的 Top-k 查询算法研究,不仅在理论上为高效算法设计提供了新思路,还在实践中推动了国产芯片技术的应用落地和生态建设。通过结合硬件特性进行定制化优化,该研究能够显著提升深度学习和大数据处理任务中的计算效率,同时推动人工智能技术的全面发展。这一研究在国家科技战略和产业实践中均具有重要价值,为我国人工智能产业的高质量发展提供了有力支撑。

\section{Top-k算法国内外研究现状}

Top-k 算法是一种从给定的数据集中找出前 k 个最大(或最小)元素的算法。这里的 k 是一个用户指定的正整数。
例如,在一个包含 100 个整数的数组中,如果 k = 10,Top - k 算法将从这个数组中找出最大(
或最小)的 10 个整数。
在大数据和人工智能快速发展的背景下,Top-k问题作为数据处理中一类重要的核心操作,广泛应用于搜索引擎、数据库管理系统、推荐系统、图神经网络等领域。如何在高性能计算环境中高效实现Top-k算法,尤其是并行算法,一直是国内外学术界和工业界关注的研究重点。
,如搜索引擎中的结果排序(找出最相关的 k 个搜索结果)、数据挖掘中的频繁项挖掘(找出出现频率最高的 k 个项)、机器学习中的特征选择(选出对模型最重要的 k 个特征)等。

传统的 Top-k 查询算法主要基于通用处理器予以设计与实现。尽管基于通用处理器设计的算法
在实现层面相对简易,然而,伴随待处理数据规模持续扩增,通用处理器已难以满足数据处理过程中
对算力的需求。因此在并行Top - $K$算法领域,将Top-k算法高效的并行实现,一直是一个研究热点。
根据实现方式,Top - k 算法可以大致分为基于排序的方法、基于选择的方法和基于概率的数据结构方法。
\begin{enumerate}
\item{基于排序的方法}:
此类别中的算法首先对整个数据集执行排序操作,随后提取前\(k\)个元素。典型的有完全排序法,像运用快速排序算法对整个数据集排序后选取前\(k\)个;还有部分排序法,例如改进的冒泡排序,仅执行\(k\)次冒泡过程以获取前\(k\)个元素。

\item{基于选择的方法}:
其中包含快速选择(Quickselect)算法,其与快速排序相似,不过仅针对划分后的特定部分开展递归操作,进而确定 Top - \(k\)元素;另外还有堆(Heap)算法,借助大小为\(k\)的最小堆(用于查找 Top - \(k\)小元素)或最大堆(用于查找 Top - \(k\)大元素)来筛选出前\(k\)个元素。在堆算法中,先将数据集中的前\(k\)个元素构建成堆,接着遍历剩余元素,若元素与堆顶元素相比更符合条件(依据查找 Top - \(k\)小或大元素而定),则替换堆顶元素并重新调整堆结构。

\item{基于概率的数据结构方法}:
比如采用 Bloom Filter + 计数的方式,先利用 Bloom Filter 初步甄别可能属于 Top - \(k\)的元素,之后通过计数手段确定实际的 Top - \(k\)元素;或者运用 Count - Min Sketch + 估计的方法,凭借对元素频率的估计来找出 Top - \(k\)元素。这些方法在应对大规模数据时,能够凭借概率数据结构在空间利用上的高效性,以近似的方式定位 Top - \(k\)元素。其中,Bloom Filter 是一种高效的概率数据结构,用于判断元素是否可能在集合中,其具有较低的空间复杂度;Count - Min Sketch 则主要用于频率估计,通过特定的参数设置来平衡估计的准确性与资源消耗。 
\end{enumerate}

\subsection{国外研究现状}
% 国外学者在Top-k并行算法的研究中侧重于算法性能优化与系统扩展性,尤其在分布式系统、流式处理和硬件加速等领域取得了显著进展。早期的研究中,Threshold Algorithm (TA) 和 No Random Access (NRA) 等算法为Top-k查询提供了基础框架\cite{fagin2001optimal}。随着分布式计算的普及,Ilyas等人(2003)提出了Ranked Join算法\cite{ilyas2003ranked},通过剪枝策略显著提高了Top-k查询在分布式环境中的效率。

% 在基于MapReduce的并行计算模型中,Chierichetti等人(2009)开发了一种高效的Top-k查询算法,该算法利用分治思想减少了跨节点的数据传输\cite{chierichetti2009fast}。近年来,基于Spark和Flink的流处理框架得到了广泛应用。Jain等人(2016)结合增量计算技术和分布式环境,提出了一种适用于动态数据流的Top-k查询优化方法\cite{jain2016apache}。

% 在硬件加速方面,Satuluri等人(2010)在GPU上实现了并行Top-k算法,通过优化CUDA内核大幅提高了矩阵计算效率\cite{satuluri2010parallel}。Aly等人(2015)进一步研究了基于FPGA的Top-k查询优化,该方法在低功耗场景中表现出显著优势\cite{aly2015accelerating}。此外,Wang等人(2016)提出了异构计算框架,结合CPU与GPU的协同计算模型,在处理大规模数据时具有更高的效率\cite{wang2016hybrid}。

% 近年来,国外学者还探索了Top-k算法与深度学习的结合。例如,Liu等人(2020)研究了神经网络稀疏激活中的Top-k选择问题,其算法广泛应用于推荐系统和搜索引擎\cite{liu2020deep}。
在 Top-k 并行算法研究的早期阶段,国外学者奠定了重要的理论与算法基础。例如,Threshold Algorithm (TA) 和 No Random Access (NRA) 等算法被提出,为 Top-k 查询提供了基础框架 \cite {fagin2001optimal}。这些算法通过设定阈值和限制随机访问等方式,初步解决了在大规模数据集中查找 Top-k 元素的基本问题,尽管其在效率和扩展性方面存在一定局限,但为后续研究提供了关键的起点。
随着分布式计算技术的逐渐普及,如何在分布式环境中高效执行 Top-k 查询成为研究热点。Ilyas 等人(2003)提出了 Ranked Join 算法 \cite {ilyas2003ranked}。该算法采用剪枝策略,在分布式环境中的多个节点间对数据进行有效的筛选与连接操作。通过提前去除大量不可能成为 Top-k 结果的数据,显著减少了数据传输量和计算量,大大提高了 Top-k 查询在分布式环境中的效率,使得分布式系统在处理大规模数据的 Top-k 查询任务时具备了更强的实用性。
在基于 MapReduce 的并行计算模型兴起后,Chierichetti 等人(2009)开发了一种高效的 Top-k 查询算法 \cite {chierichetti2009fast}。此算法巧妙地利用分治思想,将大规模数据集划分为多个子数据集,在各个子数据集上并行执行部分 Top-k 查询操作,然后再对各子结果进行合并与进一步筛选。这种方式有效地减少了跨节点的数据传输,充分发挥了 MapReduce 模型的并行处理优势,提高了整体查询效率,为在 MapReduce 框架下处理 Top-k 查询提供了一种经典的解决方案。
近年来,基于 Spark 和 Flink 的流处理框架在大数据处理领域得到了广泛应用,Top-k 算法在动态数据流场景下的研究也取得了重要进展。Jain 等人(2016)结合增量计算技术和分布式环境,提出了一种适用于动态数据流的 Top-k 查询优化方法 \cite {jain2016apache}。该方法能够实时处理不断流入的数据,在每个时间窗口内,通过增量计算快速更新 Top-k 结果,避免了对整个数据流的重新计算,有效降低了计算资源消耗,显著提升了算法在动态数据流场景下的实时性和效率,满足了诸如实时监控、网络流量分析等对数据时效性要求极高的应用需求。
为了进一步提高 Top-k 并行算法的执行速度,硬件加速成为国外研究的一个重要方向。Satuluri 等人(2010)在 GPU 上实现了并行 Top-k 算法 \cite {satuluri2010parallel}。他们通过深入优化 CUDA 内核,充分挖掘 GPU 的大规模并行计算能力,对矩阵计算进行高效处理,使得算法在处理大规模矩阵数据的 Top-k 查询时,计算效率得到大幅提高。GPU 的高并行性和快速数据处理能力为 Top-k 算法提供了强大的加速支持,尤其适用于科学计算、图像处理等对计算资源要求较高的领域。
Aly 等人(2015)则将目光投向了基于 FPGA 的 Top-k 查询优化 \cite {aly2015accelerating}。FPGA 具有可灵活编程和低功耗的特点,他们针对这些特性设计了专门的 Top-k 查询电路结构,在一些低功耗场景中,如移动设备数据处理、传感器网络数据汇聚等,该方法表现出显著优势,能够在保证一定查询效率的同时,大幅降低能耗,为在资源受限环境下的 Top-k 算法应用提供了新的思路。
Wang 等人(2016)提出了异构计算框架 \cite {wang2016hybrid},将 CPU 与 GPU 的优势相结合,构建协同计算模型。在处理大规模数据时,该框架能够根据任务特点合理分配计算资源,让 CPU 负责控制和管理任务,GPU 专注于大规模并行计算部分,充分发挥了两种硬件平台的长处,从而获得更高的整体效率,为应对复杂多样的大数据处理任务提供了一种综合性的硬件加速解决方案。
随着深度学习在人工智能领域的蓬勃发展,国外学者开始探索 Top-k 算法与深度学习的结合应用。Liu 等人(2020)研究了神经网络稀疏激活中的 Top-k 选择问题 \cite {liu2020deep}。在神经网络训练过程中,通过 Top-k 算法对神经元的激活值进行筛选,保留最具影响力的前 k 个激活值,能够有效减少计算量和模型复杂度,提高训练效率。该算法在推荐系统和搜索引擎等领域得到了广泛应用,通过对用户行为数据或搜索结果的 Top-k 筛选与分析,能够更精准地为用户提供个性化推荐和搜索结果,提升用户体验和系统性能。




\subsection{国内研究现状}
% 国内学者针对本地化需求和实际应用场景,对Top-k并行算法开展了大量研究工作,重点集中在分布式优化、流数据处理和国产硬件平台的适配上。郭建生等(2007)提出了一种分布式索引方法,通过动态分区和负载均衡技术提升了节点的资源利用效率\cite{guo2007distributed}。张敏等(2014)结合云计算环境设计了一种基于任务调度的Top-k查询算法,有效提高了算法的扩展性和计算性能\cite{zhang2014efficient}。

% 针对实时流数据处理,李明等(2018)提出了基于滑动窗口的并行Top-k算法,通过增量更新机制减少了重复计算,显著提升了实时性\cite{li2018parallel}。张伟等(2019)研究了一种动态负载均衡策略,解决了分布式环境中因数据倾斜导致的性能瓶颈问题\cite{zhang2019top}。

% 国内学者还在国产硬件平台上进行了Top-k算法的研究尝试。例如,王鹏等(2020)在飞腾GPU架构上优化了并行Top-k算法,通过硬件特性提升了矩阵分块计算的性能\cite{wang2020gpu}。赵丽等(2021)提出了一种基于FPGA的Top-k硬件加速模块,适用于低延迟、高吞吐率的应用场景,如智能监控和工业物联网\cite{zhaoli2021fpga}。

% 此外,随着隐私保护需求的增加,国内学者也开始探索分布式环境中支持加密计算的Top-k算法。例如,赵磊等(2022)研究了结合差分隐私机制的分布式Top-k查询算法,为安全计算提供了新的方向\cite{zhao2022privacy}。
国内学者针对本地化需求和实际应用场景,在分布式优化方面开展了大量研究工作。郭建生等(2007)提出了一种分布式索引方法 \cite {guo2007distributed}。该方法通过动态分区和负载均衡技术,根据数据的分布特点和节点的处理能力,动态地将数据划分为多个分区,并合理分配到各个节点上。这样不仅提高了节点的资源利用效率,避免了部分节点负载过重而其他节点闲置的情况,还能通过优化后的索引结构快速定位和访问数据,加速 Top-k 查询过程,提升了整个分布式系统的查询性能。
张敏等(2014)结合云计算环境设计了一种基于任务调度的 Top-k 查询算法 \cite {zhang2014efficient}。在云计算的分布式架构下,该算法充分考虑了不同任务的计算复杂度和数据依赖关系,通过合理的任务调度策略,将 Top-k 查询任务分配到多个虚拟机或容器中并行执行。同时,还采用了缓存机制和数据预取技术,减少了数据传输延迟和重复计算,有效提高了算法的扩展性和计算性能,使得云计算平台能够更高效地处理大规模数据的 Top-k 查询任务,为企业级大数据处理提供了有力支持。

针对实时流数据处理这一具有挑战性的领域,国内学者也取得了一系列成果。李明等(2018)提出了基于滑动窗口的并行 Top-k 算法 \cite {li2018parallel}。在处理实时流数据时,该算法引入滑动窗口机制,将数据流划分为一个个连续的时间窗口,在每个窗口内采用并行计算的方式执行 Top-k 查询。并且,通过增量更新机制,只需对新流入窗口的数据进行处理,并与上一窗口的 Top-k 结果进行合并与调整,大大减少了重复计算,显著提升了算法的实时性,能够及时响应数据流中的变化,适用于金融交易数据监控、工业生产过程实时监测等对数据时效性要求极高的应用场景。
张伟等(2019)研究了一种动态负载均衡策略 \cite {zhang2019top}。在分布式流数据处理环境中,由于数据的动态性和不均匀性,容易出现数据倾斜问题,导致部分节点负载过高而影响整体性能。该策略通过实时监测节点的负载情况和数据流量,动态地调整数据分配和任务调度,将负载较重节点上的部分任务转移到负载较轻的节点上,确保各个节点的负载相对均衡,从而提高了整个分布式系统在处理流数据的 Top-k 查询时的稳定性和效率,有效解决了因数据倾斜导致的性能瓶颈问题。

随着国产硬件技术的不断发展,国内学者积极探索 Top-k 算法在国产硬件平台上的应用与优化。王鹏等(2020)在飞腾 GPU 架构上优化了并行 Top-k 算法 \cite {wang2020gpu}。他们深入研究了飞腾 GPU 的硬件特性,如内存层次结构、计算单元性能等,对算法的数据结构和计算流程进行针对性优化。通过合理的矩阵分块计算策略,充分利用飞腾 GPU 的并行计算资源,提高了数据处理的并行度和内存访问效率,使得在国产飞腾 GPU 平台上的 Top-k 算法性能得到显著提升,为国产硬件在大数据处理领域的应用提供了技术支持和实践经验。
赵丽等(2021)提出了一种基于 FPGA 的 Top-k 硬件加速模块 \cite {zhaoli2021fpga}。该模块针对 FPGA 的可编程性和低延迟特性进行设计,通过硬件电路实现了 Top-k 算法的核心功能。在智能监控和工业物联网等低延迟、高吞吐率的应用场景中,该模块能够快速处理海量的传感器数据,实时筛选出关键信息,有效提高了系统的响应速度和数据处理能力,展示了 FPGA 在特定应用领域加速 Top-k 算法的优势和潜力。

随着数据隐私保护意识的不断增强,国内学者在分布式环境中支持加密计算的 Top-k 算法研究方面也取得了新的突破。赵磊等(2022)研究了结合差分隐私机制的分布式 Top-k 查询算法 \cite {zhao2022privacy}。该算法在分布式系统中对数据进行加密处理,使得各个节点在进行 Top-k 查询计算时,无法获取原始数据的具体信息,从而保护了数据隐私。同时,通过差分隐私技术的巧妙运用,在保证数据隐私的前提下,仍能以较高的准确率获取 Top-k 结果,为安全计算领域的 Top-k 算法应用提供了新的方向,满足了如医疗数据共享、金融数据联合分析等对数据隐私要求严格的应用场景需求。

\section{AI处理器国内外发展现状}

\subsection{国外发展现状}
在国外,AI处理器的发展长期处于世界前沿水平且呈现出蓬勃发展的态势。诸多知名企业和科研机构积极投身其中,投入大量资源进行研发创新。像英伟达,其GPU产品在AI领域占据重要地位,通过持续优化架构,如从Volta到Ampere再到Hopper架构的演进,在计算能力、能效比等关键指标上不断取得突破,广泛应用于数据中心的大规模AI训练任务以及科研领域的复杂计算场景。谷歌的TPU系列更是专门针对AI任务定制开发,从最初的TPU到后续各代升级,在性能提升、能耗降低方面成果显著,为谷歌自身的云计算服务、搜索引擎优化以及语音识别等众多业务的智能化升级提供了强劲动力。英特尔通过收购Habana Labs等战略布局,推出Gaudi等系列AI芯片,在AI处理器市场积极竞争,其芯片在特定应用场景下展现出独特优势,例如在一些对数据处理精度和吞吐量有特殊要求的企业级应用中表现出色。此外,在技术路线方面,除了主流的GPU架构,FPGA和ASIC也得到深入研究与应用。FPGA以其灵活性在一些对实时性和可重构性要求较高的AI推理应用中发挥作用,如工业自动化中的智能检测与控制环节。ASIC则凭借针对特定AI任务的定制化优势,在如智能安防领域的图像识别芯片、语音处理领域的专用芯片等方面实现了高效能与低成本的良好平衡,有力地推动了AI技术在各个行业的广泛渗透与深度应用,并且随着技术的成熟和市场需求的增长,国外AI处理器市场规模持续扩张,形成了较为完善的产业生态和市场格局,在全球AI处理器市场占据主导地位并引领着行业发展趋势。

\subsection{国内发展现状}
国内AI处理器产业在近年来奋起直追并取得了令人瞩目的成绩。以华为海思为例,其达芬奇架构AI芯片展现出强大的竞争力,通过创新的异构并行计算设计理念,在性能功耗比方面达到了国际先进水平,广泛应用于华为的智能手机终端以及服务器产品,构建起全场景AI应用能力,为华为在全球通信与智能设备市场的竞争提供了核心技术支撑。寒武纪作为专业的AI芯片设计企业,推出的思元系列芯片在智能安防监控系统、数据中心的AI计算任务等领域得到应用,并且不断探索新技术如采用Chiplet技术提升芯片性能与可扩展性,在国内AI芯片市场占据重要份额并逐步向国际市场拓展。地平线的征程系列AI芯片专注于边缘计算场景,特别是在智能汽车领域,为车辆的自动驾驶辅助系统提供高效的AI计算能力,助力国内智能汽车产业的快速发展。在产业生态建设方面,政府积极出台一系列扶持政策,包括设立专项产业投资基金提供资金支持、给予税收优惠政策鼓励企业创新研发、推动高校与企业合作培养专业人才等,为AI处理器产业营造了良好的政策环境。上下游企业之间的合作也日益紧密,从芯片设计、晶圆制造、封装测试到软件算法开发以及终端应用集成,逐步形成完整的产业链条,促进了产业协同发展。然而,不可忽视的是,国内AI处理器产业仍面临一些挑战。在芯片设计技术的高端层面,与国外顶尖水平相比在某些复杂算法处理能力和架构优化细节上尚有差距。在制造工艺方面,由于受到高端光刻机等关键设备进口限制,芯片制程工艺的先进性难以与国外领先企业相媲美,这在一定程度上影响了芯片的性能、成本和量产规模。同时,在资金投入的持续性和人才储备的丰富度上,与国外大型企业和成熟科研体系相比略显不足,但随着国内市场对AI技术需求的爆发式增长以及外部技术封锁压力下的自主创新动力增强,国内AI处理器产业正处于快速发展的关键时期,蕴含着巨大的发展潜力和突破机遇,有望在未来逐步缩小与国外的差距并在国际市场上占据更为重要的地位。


\section{论文主要内容及章节安排}
\subsection{论文主要工作}
本文重点研究 Top - k 算法在国产AI平台上的实现与性能调优后的算子的精度和性能,
主要研究内容包括:

(1) 分析Top-k算法,并介绍国产AI处理器的存储模型和异构编程模型,以此作为后面研究国产AI处理器数据级并行化和异步并行化的基础。

(2) 针对国产AI处理器对Top-k算子进行并行实现。

(3) 通过从计算效率和访存效率两方面对Top-k算子进行性能调优。

(4) 对Top-k算子的并行化方案以及性能调优后的方案进行精度与性能上的测试,并在深度学习框架上进行适配,进行集成测试
以验证Top-k算子的可用性。

\subsection{论文结构}

本文共六个章节,各章内容组织如下: 

第一章为绪论,首先介绍本文的研究背景及意义,接着对Top-k算法的国内外研究现状和应用现状
以及 AI 处理器的国内外发展现状进行详细介绍,
最后对本文研究内容和章节安排情况进行简要概括。

第二章为相关技术背景,首先介绍了radix-select Top-k算法的原理,接着对国产AI处理器的
编程模型和内存模型进行介绍。最后介绍了面向国产AI处理器的深度学习框架——pytorch。


第三章为Top-k算子设计与实现,本章节主要基于radix-slect算法,设计了Top-k算子,
并且根据Top-k算子的输入规模和国产AI处理器的体系结构,设计了两种不同的并行方案。
本章主要分为四部分,第一部分首先介绍了算子的计算流程,在此详细讲解了Top-k算子工作过程中的
几个重要阶段。而后根据其计算流程,详细介绍了主机端的主要任务。最后依据radix-select Top-k算法原理
和国产AI处理器的体系结构,详细的描述了两种并行实现方案和实现过程。

第四章为Top-k算子的性能优化。
首先介绍算子的优化策略,接着分别从计算效率和访存效率两方面,
对Top-k算子的优化方案进行全面讲解。
 
 第五章为实验测试与分析,本章在多核深度学习处理器上,对三四章节所实现和优化的Top-k算子进行全面的精度测试与性
 能评估,并集成到深度学习框架中,验证Top-k算子的可用性。 
 
 第六章为总结与展望,本章对全文的工作进行整理
 总结,并对本文方法中存 在的不足进行分析,然后提出后续改进的方向。
